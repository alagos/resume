%-------------------------------------------------------------------------------
%	SECTION TITLE
%-------------------------------------------------------------------------------
\cvsection{Presentation}


%-------------------------------------------------------------------------------
%	CONTENT
%-------------------------------------------------------------------------------
\begin{cventries}

%---------------------------------------------------------
  \cventry
    {\href{https://codenuble.cl/}{\underline{Code Ñuble}}} % Institution
    {\href{https://alagos.github.io/elixir-talk}{\underline{Functional programming with Elixir}}} % Role
    {Chillan, Chile (remote)} % Location
    {Jun. 2019} % Date(s)
    {
      \begin{cvitems} % Description(s)
        \item {
          Origin of Elixir Language, some of their features and a quick comparison with Ruby about its functional paradigm.
        }
      \end{cvitems}
    }

%---------------------------------------------------------
  \cventry
    {\href{http://www.ubiobio.cl/}{\underline{Bio-Bio University}}} % Institution
    {\href{https://alagos.github.io/trabajo-remoto}{\underline{Trabajo Remoto (Remote work)}}} % Institution
    {Chillan, Chile} % Location
    {Nov. 2014} % Date(s)
    {
      \begin{cvitems} % Description(s)
        \item {
          My experience working as remote freelancer. Explaining advantages/disadvantages and giving some tips about how to start with this work modality.
        }
      \end{cvitems}
    }

%---------------------------------------------------------
  \cventry
    {\href{https://www.meetup.com/es-ES/starsconf/events/85184462/}{\underline{dynLang Chile (meetup.com)}}} % Institution
    {\href{https://alagos.github.io/optimizando_rails}{\underline{Optimizando Rails (Optimizing Rails)}}} % Institution
    {Santiago, Chile} % Location
    {Nov. 2012} % Date(s)
    {
      \begin{cvitems} % Description(s)
        \item {
          Talk about different techniques to optimize a Rails application, including code profiling, query optimization, caching and asset management.
        }
      \end{cvitems}
    }

%---------------------------------------------------------
  \cventry
    {\href{http://santiago.flisol.cl/}{\underline{FLISOL 2010 (Latin American Festival of Free Software)}}} % Organization
    {“Radiografía de un sitio web” (A Website's Radiography)} % Role
    {Santiago, Chile} % Location
    {May. 2010} % Date(s)
    {
      \begin{cvitems} % Description(s)
        \item {
          Giving good practice tips regarding the use of HTML/CSS/JS. Also an introduction about HTML5 and all the improvements it brings to the web.
        }
      \end{cvitems}
    }

%---------------------------------------------------------
\end{cventries}
